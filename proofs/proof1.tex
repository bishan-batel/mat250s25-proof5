\begin{question}
	Prove that $q \ge p$ ($P_1$).

	\begin{proof}

		Assuming $\neg P_1 \implies \exists U \subseteq \R^n, V \subseteq U : \dim V > \dim U$.


		Let $\set{ \vec v_1, \vec v_2, \cdots, \vec v_q }$ be a basis of $V$.


		Assuming $q>p$, we can take $p$ vectors from the set $\set{\vec v_1, \vec v_2, \cdots, \vec v_p, \cdots, \vec v_q}$  to form a basis of $U$ as the set $\set{\vec v_1, \vec v_2, \cdots, \vec v_p}$.

		Because $V \subseteq U$, all vectors must in both sets must be inside $U$.

		$$V \subseteq U \implies \forall i\, v_i \in U$$

		Forming a linearly independent set that spans $U$ from $\set{\vec v_1, \vec v_2, \cdots, \vec v_p}$, the vector $\vec v_q$ is not inside this set but is an element in $U$. This must mean there
		exists a linear combination between the set and some vector $\vec x$ such that it equals $\vec v_q$.

		$$ \neg P_1 \implies \exists \vec x : \mat{\vec v_1 & \cdots & \vec v_p}x = \vec v_q $$

		This imposes a contradiction to $\mathcal L_2$, as the set $\set{\vec v_1, \vec v_2, \cdots, \vec v_p, \cdots, \vec v_q}$ is linearly independent, therefore $P_1$ must be true.
	\end{proof}
\end{question}
