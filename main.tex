\documentclass{exam}

\usepackage{amsfonts}
\usepackage{amssymb}
\usepackage{mathtools}
\usepackage{braket}
\usepackage{forloop}
\usepackage{amsthm}
\usepackage[backend=biber]{biblatex}

\theoremstyle{plain}
\newtheorem{assumption}{Assumption}

\theoremstyle{definition}
\newtheorem{definition}{Definition}

\newtheorem{lemma}{Lemma}


% huge align
\newcommand{\ha}[1]{{\huge{\begin{align*}#1\end{align*}}}}



% P & S Are excluded 
% \newcommand{\A}[0]{{\mathbb A}}
% \newcommand{\B}[0]{{\mathbb B}}
% \newcommand{\C}[0]{{\mathbb C}}
% \newcommand{\D}[0]{{\mathbb D}} 
% \newcommand{\E}[0]{{\mathbb E}} 
% \newcommand{\F}[0]{{\mathbb F}} 
% \newcommand{\G}[0]{{\mathbb G}} 
% \newcommand{\H}[0]{{\mathbb H}} 
% \newcommand{\I}[0]{{\mathbb I}} 
% \newcommand{\J}[0]{{\mathbb J}} 
% \newcommand{\K}[0]{{\mathbb K}} 
% \newcommand{\L}[0]{{\mathbb L}} 
% \newcommand{\M}[0]{{\mathbb M}} 
\newcommand{\N}[0]{{\mathbb N}}
% \newcommand{\O}[0]{{\mathbb O}} 
% \newcommand{\P}[0]{{\mathbb P}} 
\newcommand{\Q}[0]{{\mathbb Q}}
\newcommand{\R}[0]{{\mathbb R}}
% \renewcommand{\S}[0]{{\mathbb S}}
% \newcommand{\T}[0]{{\mathbb T}}
% \newcommand{\U}[0]{{\mathbb U}}
% \newcommand{\V}[0]{{\mathbb V}}
% \newcommand{\W}[0]{{\mathbb W}}
% \newcommand{\X}[0]{{\mathbb X}}
% \newcommand{\Y}[0]{{\mathbb Y}}
\newcommand{\Z}[0]{{\mathbb Z}}

% Calculus
% \newcommand{\d}[0]{{\mathrm{d}}}
\newcommand{\deriv}[2]{ \frac{ \d{#1} }{ \d{#2} } }
\newcommand{\pderiv}[2]{ \frac{ \partial{#1} }{ \partial{#2} } }

\newcommand{\nderiv}[3]{ \frac{ \d^{#1}{#2} }{ \d{#3}^{#1} } }
\newcommand{\npderiv}[3]{ \frac{ \partial^{#1}{#2} }{ \partial{#3}^{#1} } }

% Linear Algebra 
\renewcommand{\vector}[1]{ \overrightarrow{#1} }
\newcommand{\vecn}[1]{ {\hat #1} }

\newcommand{\mat}[1]{{ \begin{bmatrix} #1 \end{bmatrix} }}
\newcommand{\mats}[1]{{ \ba{\begin{smallmatrix} #1 \end{smallmatrix}} }}

\newcommand{\pmat}[1]{{ \begin{pmatrix} #1 \end{pmatrix} }}
\newcommand{\pmats}[1]{{ \pa{\begin{smallmatrix} #1 \end{smallmatrix}} }}

\newcommand{\emat}[1]{{ \begin{ematrix} #1 \end{ematrix} }}
\newcommand{\emats}[1]{{ \begin{smallmatrix} #1 \end{smallmatrix} }}

\newcommand{\vmat}[1]{{ \begin{vmatrix} #1 \end{vmatrix} }}

\newcommand{\rowechelon}[1]{{
			\left[\begin{array}{ccc|c} #1 \end{array}\right]
		}}

\newcommand{\augmented}[2]{{
			\left[\begin{array}{#1} #2 \end{array}\right]
		}}

% Generic Notatino
\newcommand{\paren}[1]{{ \left(#1\right) }}

\newcommand{\pa}[1]{{ \left(#1\right) }}
\newcommand{\ba}[1]{{ \left[#1\right] }}


\newcommand{\llet}[0]{ {\text{let } } }
\newcommand{\undefined}[0]{ {\text{undefined.} } }

\newcommand{\op}[1]{ {\operatorname{#1} } }

\newcommand{\brt}[2]{ {\root {#1} \of {#2} } }

\newcommand{\proj}[1]{ { \op{proj}_{#1} }}
\newcommand{\projperp}[1]{ { \op{proj}_{#1\perp} } }

\newcommand{\norm}[1]{{ {\left\lVert #1 \right\rVert} }}
\newcommand{\norms}[1]{{ {\lVert #1 \rVert} }}

% CS 
\newcommand{\hex}[1]{{ \pa{\mathrm{#1}}_{16} }}
\newcommand{\bin}[1]{{ \pa{#1}_{2} }}
\newcommand{\binb}[2]{{ \pa{#1}^{#2}_{2} }}
\newcommand{\dec}[1]{{ \pa{#1}_{10} }}


\newcommand{\true}[0]{{ \mathrm{true} }}
\newcommand{\false}[0]{{ \mathrm{false} }}

\renewcommand{\ba}[1]{{ \left[ {#1} \right] }}

\newcommand{\ceil}[1]{{ \left\lceil {#1} \right\rceil }}
\newcommand{\floor}[1]{{ \left\lfloor {#1} \right\rfloor }}

\newcommand{\ang}[1]{{ \left\langle {#1} \right\rangle }}

\newcommand{\transpose}[1]{ { {#1}^{\intercal} } }





\addbibresource{references.bib}

\begin{document}

\title{MAT250 Proof 5}
\author{Kishan S Patel}
\maketitle

\renewcommand{\qedsymbol}{QED}

\begin{lemma}[$\mathcal L_1$]
	For any linear subspace $U$, a basis of $U$ constructed from some set $S$ of linearly independent vectors in $U$ must have a cardinality of $\dim U$.

	$$ \forall n\in \N,U \subseteq \R^n  : U = \op{span}\pa{  \left\{  \vec u_1, \vec u_2, \cdots, \vec u_{\dim U}  \right\} } $$
\end{lemma}

\begin{lemma}[$\mathcal L_2$]
	For any linearly independent set of vectors $\set{\vec v_1, \cdots, \vec v_n}$, for all $v_i$ there does not exist a linear combination with all other vectors ($v_i : i \neq j$) that sums to $v_i$.
\end{lemma}


Let $\R^n$ be a Euclidean space, and let $U, V\subseteq \R^n$ be subspaces such that $V \subseteq U$

Let $p = \dim U$ and $q=\dim V$.


\begin{questions}
	\begin{question}
	Prove that $q \ge p$ ($P_1$).

	\begin{proof}

		Assuming $\neg P_1 \implies \exists U \subseteq \R^n, V \subseteq U : \dim V > \dim U$.


		Let $\set{ \vec v_1, \vec v_2, \cdots, \vec v_q }$ be a basis of $V$.


		Assuming $q>p$, we can take $p$ vectors from the set $\set{\vec v_1, \vec v_2, \cdots, \vec v_p, \cdots, \vec v_q}$  to form a basis of $U$ as the set $\set{\vec v_1, \vec v_2, \cdots, \vec v_p}$.

		Because $V \subseteq U$, all vectors must in both sets must be inside $U$.

		$$V \subseteq U \implies \forall i\, v_i \in U$$

		Forming a linearly independent set that spans $U$ from $\set{\vec v_1, \vec v_2, \cdots, \vec v_p}$, the vector $\vec v_q$ is not inside this set but is an element in $U$. This must mean there
		exists a linear combination between the set and some vector $\vec x$ such that it equals $\vec v_q$.

		$$ \neg P_1 \implies \exists \vec x : \mat{\vec v_1 & \cdots & \vec v_p}x = \vec v_q $$

		This imposes a contradiction to $\mathcal L_2$, as the set $\set{\vec v_1, \vec v_2, \cdots, \vec v_p, \cdots, \vec v_q}$ is linearly independent, therefore $P_1$ must be true.
	\end{proof}
\end{question}

	\begin{question}
	If $AB$ is surjective as a transformation, then so is $A$.

	\begin{proof}
		\begin{align*}
			\forall z \in Z, \exists x \in X            & : (f\circ g)(x) = z \\
			f(g(x))                                     & = z                 \\
			\llet y\in Y                                & = g(x)              \\
			f(y)                              = f(g(x)) & = z                 \\
		\end{align*}
		$$ \therefore f \text{ is surjective.} $$
	\end{proof}
\end{question}

\end{questions}

\end{document}
