\documentclass{exam}

\usepackage{amsfonts}
\usepackage{amssymb}
\usepackage{mathtools}
\usepackage{braket}
\usepackage{forloop}
\usepackage{amsthm}
\usepackage[backend=biber]{biblatex}

\theoremstyle{plain}
\newtheorem{assumption}{Assumption}

\theoremstyle{definition}
\newtheorem{definition}{Definition}

\newtheorem{lemma}{Lemma}

\input{preamble.sty}

\addbibresource{references.bib}

\begin{document}

\title{MAT250 Proof 5}
\author{Kishan S Patel}
\maketitle

\renewcommand{\qedsymbol}{QED}

\begin{lemma}[$\mathcal L_1$]
	For any linear subspace $U$, a basis of $U$ constructed from some set $S$ of linearly independent vectors in $U$ must have a cardinality of $\dim U$.

	$$ \forall n\in \N,U \subseteq \R^n  : U = \op{span}\pa{  \left\{  \vec u_1, \vec u_2, \cdots, \vec u_{\dim U}  \right\} } $$
\end{lemma}

\begin{lemma}[$\mathcal L_2$]
	For any linearly independent set of vectors $\set{\vec v_1, \cdots, \vec v_n}$, for all $v_i$ there does not exist a linear combination with all other vectors ($v_i : i \neq j$) that sums to $v_i$.
\end{lemma}


Let $\R^n$ be a Euclidean space, and let $U, V\subseteq \R^n$ be subspaces such that $V \subseteq U$

Let $p = \dim U$ and $q=\dim V$.


\begin{questions}
	\begin{question}

	If $AB$ is one-to-one (injective) as a linear transformation, then so is $B$ ($P_1$).


	\begin{proof}


		$$
			\begin{align*}
				\llet f & : Y       \to 	 Z                 \\
				f       & : \vec x  \mapsto 	 A\vec x       \\
				\\
				\llet g & : X       \to 	 Y                 \\
				g       & : \vec x        \mapsto 	 B\vec x
			\end{align*}
		$$

		$f\circ g$ is injective because $f\circ g = AB$.

		$$
			\begin{align*}
				\llet P_{2}	 =		\not\exists\, x_0, x_1\in X : \, (f\circ g)(x_0) = (f\circ g)(x_1) \wedge x_0 \neq x_1
			\end{align*}
		$$

		\begin{align*}
			\neg P_1  \implies & \exists x_0, x_1 \in X  : \, g(x_0)           = g(x_1) \wedge x_0 \neq x_1                       \\
			\implies           & \exists x_0,x_1\in X                 : \, (f\circ g)(x_0)  = (f\circ g)(x_1) \wedge x_0 \neq x_1
		\end{align*}
		$$ P_2 \wedge	\neg P_1 \implies  \neg P_2 \therefore P_1 \text{ is true.} $$
	\end{proof}
\end{question}

	\begin{question}
	If $q=p$ then $V=U$ ($P_2$).

	\begin{proof}
		Let $\vec u \in U$. Because $V\subseteq U$ and $\dim U = \dim V$, any basis $\set{\vec v_1, \cdots, \vec v_q}$ of $V$ is also a basis of $U$ as all elements in the set are elements of $U$.

		Because any basis of $V$ is a basis of $U$, $\vec u$ can be written as a linear combination of the basis vectors of $V$, therefore $\vec u \in V$.

		\begin{align*}
			u \in V \wedge u \in U \implies U \subseteq V     \\
			V \subseteq U \wedge U \subseteq V \implies U = V \\
		\end{align*}
	\end{proof}
\end{question}

\end{questions}

\end{document}
